%%%%%%%%%%%%%%%%%%%%%%%%%%%%%%%%%%%%%%%%%
% FRI Data Science_report LaTeX Template
% Version 1.0 (28/1/2020)
% 
% Jure Demšar (jure.demsar@fri.uni-lj.si)
%
% Based on MicromouseSymp article template by:
% Mathias Legrand (legrand.mathias@gmail.com) 
% With extensive modifications by:
% Antonio Valente (antonio.luis.valente@gmail.com)
%
% License:
% CC BY-NC-SA 3.0 (http://creativecommons.org/licenses/by-nc-sa/3.0/)
%
%%%%%%%%%%%%%%%%%%%%%%%%%%%%%%%%%%%%%%%%%


%----------------------------------------------------------------------------------------
%	PACKAGES AND OTHER DOCUMENT CONFIGURATIONS
%----------------------------------------------------------------------------------------
\documentclass[fleqn,moreauthors,10pt]{ds_report}
\usepackage[english]{babel}

\graphicspath{{fig/}}




%----------------------------------------------------------------------------------------
%	ARTICLE INFORMATION
%----------------------------------------------------------------------------------------

% Header
\JournalInfo{Machine Learning for Data Science, 2025}

% Article title
\PaperTitle{
Multinomial and Ordinal Logistic Regression Implementations
}

% Authors (student competitors) and their info
\Authors{Teodora Taleska}

%----------------------------------------------------------------------------------------

\begin{document}

% Makes all text pages the same height
%\flushbottom
%
\raggedbottom

% Print the title and abstract box
\maketitle

% Removes page numbering from the first page
\thispagestyle{empty} 

%----------------------------------------------------------------------------------------
%	ARTICLE CONTENTS
%----------------------------------------------------------------------------------------

%\section{Introduction}
%In this project, we implemented two types of logistic regression models: Multinomial Logistic Regression and Ordinal Logistic Regression. Both models were built as classes and were fitted using maximum likelihood estimation via optimization methods, specifically L-BFGS-B, provided by the \texttt{scipy} library.

\section{Multinomial Logistic Regression}
The primary task in this model is to model the probabilities of each class given the input features.

\subsection{Model Structure}
In the implementation, the model is fitted using maximum likelihood estimation. We begin by calculating the log-likelihood of the model parameters, which is given by:

\[
\mathcal{L}(\beta) = \sum_{i=1}^{n} \log(P(y_i|X_i, \beta))
\]

where \(P(y_i|X_i, \beta)\) is the probability of class \(y_i\) for the \(i\)-th sample, computed using the \textit{softmax} function.

The probabilities are computed via the softmax function:

\[
P(y_i = j | X_i) = \frac{\exp(\beta_j^T X_i)}{\sum_{k=1}^m \exp(\beta_k^T X_i)}
\]

where \(X_i\) is the feature vector, \(\beta_j\) is the coefficient for class \(j\), and \(m\) is the number of classes. This ensures that the sum of the predicted probabilities for all classes is 1 for each observation.

\subsection{Optimization and Coefficient Estimation}
The model parameters, including the weights (\(\beta\)) and intercepts, are estimated using the L-BFGS-B optimization algorithm. The optimization maximizes the log-likelihood function by updating the parameters iteratively. The gradient of the log-likelihood with respect to the parameters is computed analytically to efficiently guide the optimization process.

\[
\text{Gradient} = \frac{\partial \mathcal{L}(\beta)}{\partial \beta}
\]

The gradient is calculated using the chain rule, with the softmax function applied to the logits (the linear combination of features and weights). This allows us to adjust the parameters in the direction of maximum likelihood.

%\subsection{Testing the Model}
%The MLR model was tested using synthetic data with multiple classes. The model's predictions were evaluated based on the predicted class probabilities, and the gradients were numerically checked for correctness using \texttt{scipy.check\_grad}. This ensures that the implementation is numerically stable and the gradient computation is accurate.

\section{Ordinal Logistic Regression}
This model is used for classification tasks where the target variable has ordered categories (ordinal data). Unlike multinomial logistic regression, where the classes are independent, ordinal logistic regression assumes an inherent order between the categories.

\subsection{Model Structure}
In this model, we compute the probabilities of each class using a cumulative distribution function (CDF) based on the thresholds between classes. The model computes the probability of a sample belonging to a certain class using the logistic CDF:

\[
P(y_i \leq j | X_i) = \frac{1}{1 + \exp(-(X_i^T \beta + \delta_j))}
\]

where \(\delta_j\) are the thresholds between classes. The final probability for class \(j\) is then the difference between the cumulative probabilities for class \(j\) and class \(j-1\).

\subsection{Threshold Calculation and Deltas}
The thresholds between the classes are determined by cumulative deltas, which are optimized during the fitting process. The deltas control the shift between thresholds. We start by setting initial deltas and compute the thresholds based on the cumulative sum of deltas. These thresholds are then used to compute the class probabilities for each sample.

\[
t_j = \sum_{k=1}^{j} \delta_k
\]

The model optimizes these deltas to maximize the log-likelihood, with the objective being the same as in multinomial logistic regression: maximizing the probability of the correct class labels given the input features.

\subsection{Optimization and Coefficient Estimation}
The fitting procedure for ordinal logistic regression involves optimizing both the coefficients (\(\beta\)) and deltas using the L-BFGS-B algorithm. The negative log-likelihood function is minimized:

\[
\mathcal{L}(\beta, \delta) = -\sum_{i=1}^{n} \log(P(y_i | X_i, \beta, \delta))
\]

where the probabilities are computed based on the thresholds and coefficients, and the deltas are constrained to be positive.

%\subsection{Testing the Model}
%The ordinal logistic regression model was tested with synthetic data using four ordinal classes. The model’s predictions were evaluated by checking whether the predicted class probabilities sum up to one. The gradient and optimization performance were assessed in a similar manner to the multinomial model. Moreover, the thresholds were checked to ensure they are in increasing order.

%------------------------------------------------

%\section*{Discussion}


%------------------------------------------------


%----------------------------------------------------------------------------------------
%	REFERENCE LIST
%----------------------------------------------------------------------------------------
%\bibliographystyle{unsrt}
%\bibliography{report}


\end{document}